\section{Preliminaries}

An article by Suzuki and Tanaka \cite{suzuki2021}, published in 2021, proposes that stress exposure is associated with reduced activity in the \textbf{ventromedial prefrontal cortex (vmPFC)} and heightened activity in limbic regions, particularly the amygdala. These brain regions are believed to be central to the mechanisms of emotional regulation.

\begin{quote}
"Exposure is associated with reduced activity in the vmPFC and increased activity in the amygdala and other limbic regions, which is thought to underlie the shift from flexible, goal-directed behaviour to more rigid, habitual and emotionally driven responses."
\end{quote}

Another influential study by Urry et al. \cite{Urry2006}, published in 2006, provides evidence that the amygdala and the vmPFC are \emph{inversely coupled} during emotional regulation. Such a coupling pattern is particularly observable in individuals experiencing trauma or heightened emotional reactivity.

\begin{quote}
"Amygdala and ventromedial prefrontal cortex are inversely coupled during emotion regulation."
\end{quote}
\begin{quote}
"Greater vmPFC activity is associated with reduced amygdala activity, supporting the model that vmPFC exerts top-down control over emotional responses."
\end{quote}

These and other studies suggest a robust connection and mapping between emotional-states and unconscious thoughts. In classical literature, evidences are still present supporting this hypothesis.
