\documentclass[9pt]{article}

% Packages
\usepackage[margin=1in]{geometry}
\usepackage{setspace}
\usepackage{amsmath, amssymb}
\usepackage{hyperref}

% Title, author, and affiliation
\title{On Emotional Resonance Hypothesis}
\author{Jayant Sharma\\
\small Independent Researcher in Theoretical Psychology}

\date{\today}

\begin{document}

\maketitle

\begin{abstract}
The Emotional Resonance Mapping (ERM) Hypothesis posits that an individual’s unconscious cognitive processes and their contemporaneous affective states exhibit a significant degree of mutual resonance. This dynamic interplay is hypothesized to be sufficiently robust—particularly during heightened emotional episodes—to permit bidirectional or correlative mapping between subconscious ideation and real-time emotional valence. The theoretical foundation of this hypothesis is supported by emerging empirical findings within affective neuroscience and psychodynamic research.

In the present study, we undertake a systematic examination of the ERM framework and elucidate its integrative connections with the domains of traumatology, psychotraumatology, and contemporary dream theory. We aim to demonstrate how ERM not only aligns with established trauma models but may also offer novel insights into the encoding and manifestation of unresolved psychological conflict in both waking cognition and dream content.
\end{abstract}

\section{Preliminaries}

An article by Suzuki and Tanaka \cite{suzuki2021}, published in 2021, proposes that stress exposure is associated with reduced activity in the \textbf{ventromedial prefrontal cortex (vmPFC)} and heightened activity in limbic regions, particularly the amygdala. These brain regions are believed to be central to the mechanisms of emotional regulation.

\begin{quote}
"Exposure is associated with reduced activity in the vmPFC and increased activity in the amygdala and other limbic regions, which is thought to underlie the shift from flexible, goal-directed behaviour to more rigid, habitual and emotionally driven responses."
\end{quote}

Another influential study by Urry et al. \cite{Urry2006}, published in 2006, provides evidence that the amygdala and the vmPFC are \emph{inversely coupled} during emotional regulation. Such a coupling pattern is particularly observable in individuals experiencing trauma or heightened emotional reactivity.

\begin{quote}
"Amygdala and ventromedial prefrontal cortex are inversely coupled during emotion regulation."
\end{quote}
\begin{quote}
"Greater vmPFC activity is associated with reduced amygdala activity, supporting the model that vmPFC exerts top-down control over emotional responses."
\end{quote}

These and other studies suggest a robust connection and mapping between emotional-states and unconscious thoughts. In classical literature, evidences are still present supporting this hypothesis.


\bibliographystyle{plain}
\bibliography{document}

\end{document}
