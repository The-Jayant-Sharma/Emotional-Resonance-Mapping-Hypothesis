\documentclass[9pt]{article}

% Packages
\usepackage[margin=1in]{geometry}
\usepackage{setspace}
\usepackage{amsmath, amssymb}
\usepackage{hyperref}

% Title, author, and affiliation
\title{On Emotional Resonance Hypothesis}
\author{Jayant Sharma\\
\small Independent Researcher in Theoretical Psychology}

\date{\today}

\begin{document}

\maketitle

\begin{abstract}
The Emotional Resonance Mapping (ERM) Hypothesis posits that an individual’s unconscious cognitive processes and their contemporaneous affective states exhibit a significant degree of mutual resonance. This dynamic interplay is hypothesized to be sufficiently robust—particularly during heightened emotional episodes—to permit bidirectional or correlative mapping between subconscious ideation and real-time emotional valence. The theoretical foundation of this hypothesis is supported by emerging empirical findings within affective neuroscience and psychodynamic research.

In the present study, we undertake a systematic examination of the ERM framework and elucidate its integrative connections with the domains of traumatology, psychotraumatology, and contemporary dream theory. We aim to demonstrate how ERM not only aligns with established trauma models but may also offer novel insights into the encoding and manifestation of unresolved psychological conflict in both waking cognition and dream content.
\end{abstract}

\section{Preliminaries}
An article by Suzuki, Y. and Tanaka, S.C., \cite{suzuki2021} published in 2021, proposes the idea that stress exposure is associated with reduced activity in \textbf{ventromedial prefrontal cortex} and domination in limbic regions and amygdala which are assumed to be neccesary for emotional Regulation.

\begin{quote}
"Exposure is associated with reduced activity in the vmPFC and increased activity in the amygdala and other limbic regions, which is thought to underlie the shift from flexible, goal-directed behaviour to more rigid, habitual and emotionally driven responses."
\end{quote} 

Another study by Urry, Heather L. and van Reekum, Carien M. and Johnstone, Tom and Kalin, Ned H. and Thurow, Marchell E. and Schaefer, Hillary S. and Jackson, Cory A. and Frye, Corrina J. and Greischar, Lawrence L. and Alexander, Andrew L. and Davidson, Richard J. \cite{Urry2006} published in 2006, acknowledges that amygala and ventromedial prefrontal cortex are inversely coupled during emotional Regulation. Such emotional-regulation states are observable in traumatic episodes or even some hyper-active states.

\begin{quote}
"Amygdala and ventromedial prefrontal cortex are inversely coupled during emotion regulation."
\end{quote}
\begin{quote}
"Greater vmPFC activity is associated with reduced amygdala activity, supporting the model that vmPFC exerts top-down control over emotional responses."
\end{quote}

These researches along with numerous others shape the idea that \emph{emotional Regulation is inversely coupled with the activities related to Prefrontal Cortex.}
This impairment results in lesser ability to take logical decisions in hyper-emotional states. This, leads us to investigate if \emph{thoughts which do seem to restrict control of the person are deeply intertwined with the real-time emotional state of a person.}


\bibliographystyle{plain}
\bibliography{document}

\end{document}
